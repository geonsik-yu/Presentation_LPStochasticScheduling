\documentclass{beamer}
%
% Choose how your presentation looks.
%
% For more themes, color themes and font themes, see:
% http://deic.uab.es/~iblanes/beamer_gallery/index_by_theme.html
%
\mode<presentation>
{
  \usetheme{default}      % or try Darmstadt, Madrid, Warsaw, ...
  \usecolortheme{default} % or try albatross, beaver, crane, ...
  \usefonttheme{serif}  % or try serif, structurebold, ...
  \setbeamertemplate{navigation symbols}{}
  \setbeamertemplate{caption}[numbered]
  
} 

\usepackage[english]{babel}
\usepackage[utf8x]{inputenc}
\usepackage{amsthm} % Added
\usepackage[mathscr]{eucal}
\usepackage{ragged2e}



\title[Your Short Title]{Approximation in Stochastic Shceduling:\\The Power of LP-Based Priority Policies}
\author{Rolf H. M\"{o}hring, Andreas S, Schulz and Marc Uetz}
\institute{Slides: Geonsik Yu}
\begin{document}

\begin{frame}
  \titlepage
\end{frame}

% Uncomment these lines for an automatically generated outline.
%\begin{frame}{Outline}
%  \tableofcontents
%\end{frame}

\section{Notations}
\begin{frame}{Notations}
    \begin{itemize}
        \justifying
        \item $\pmb{J=\{1,2,\dots,n\}}$ be a set of jobs that have to be nonpreemptively scheduled on $m$ identical parallel machines.
        \vspace{0.2cm}
        \item $\pmb{w_j}$ denotes each job $j$'s non-negative weight.
        \vspace{0.2cm}        
        \item $\pmb{C_j}$ denotes the completion time of job $j$.
        \vspace{0.2cm}
        \item $\pmb{r_j}$ denotes the release date of job $j$.
        \vspace{0.2cm}
        \item $\pmb{p=(p_1,\dots,p_n)}$ represents a vector of random variables where $\pmb{p_j}$ denotes the random variable for the processing time of job $j$ which is not known in advance.
        \vspace{0.2cm}
        \item Our goal is to minimize the total weighted completion time ($\pmb{\text{Minimize }\sum\limits_{j\in J}w_jC_j}$).
    \end{itemize}
\end{frame}

\section{Notations}
\begin{frame}{Notations: Policies($\Pi$)}
    \begin{itemize}
        \justifying
        \item \textbf{\textit{Scheduling policy ($\Pi$)}}: An on-line(dynamic) allocation criteria of jobs to machines.
        \vspace{0.6cm}
        \item \textbf{\textit{Priority policy}}: If at any time $t$ a maximal number of available jobs is scheduled according to a given priority order on the set of jobs. E.g. LEPT, SEPT, and WSEPT.
        \vspace{0.6cm}
        \item \textbf{\textit{Job-based priority policy}}: When release dates or precedence constraints are additionally given, this policy enforces jobs with low priority to be scheduled only if all jobs with higher priority have already been started. 
    \end{itemize}
\end{frame}


\section{Problem Definition}
\begin{frame}{Problem Definition}
    \begin{itemize}
        \justifying
        \item If $E[C_j^\Pi]$ denotes the expected completion time of job $j$ when scheduling according to policy $\Pi$, our problem can be formulated as follows:
        \vspace{0.8cm}
        \begin{equation}
        \text{Minimize } \bigg\{\sum\limits_{j\in J} w_j E[\pmb{C}_j^\Pi] \bigg\vert \text{ } \Pi \text{ policy} \bigg\}    \end{equation}
        \vspace{0.8cm}
        \item Let $Z^{OPT}$ be the optimum value of (1).
    \end{itemize}
\end{frame}

\begin{frame}{Problem Definition}
    \begin{itemize}
        \justifying
        \item (1) can be rewritten as follows:
        \vspace{0.8cm}
        \begin{equation}
        \text{Minimize } \bigg\{\sum\limits_{j\in J} w_j E[C_j] \bigg\vert \text{ } C \in \mathscr{C} \bigg\}
        \end{equation}
        \vspace{0.8cm}
        \item[] Where $\mathscr{C}:=\{(E[C_1^{\Pi}],E[C_2^{\Pi}],\dots,E[C_n^{\Pi}]) \vert \Pi \text{ policy} \} \in \mathbb{R}_{+}^n$ denotes the \textit{performance space}.
    \end{itemize}
\end{frame}

\section{LP-Based Approximation}
\begin{frame}{LP-Based Approximation}
    \begin{itemize}
        \justifying
        \item Since one cannot completely characterize the performance space($\mathscr{C}$), we approximate $\mathscr{C}$ by a polyhedron $\mathscr{P}$ which is defined by valid inequalities for $\mathscr{C}$. Thus, $\pmb{\mathscr{C} \subseteq \mathscr{P}}$. We then solve the following LP relaxation:
        \vspace{0.8cm}
        \begin{equation}
        \text{Minimize } \bigg\{\sum\limits_{j\in J} w_j E[C_j] \bigg\vert \text{ } C \in \mathscr{P} \bigg\}
        \end{equation}
        \vspace{0.8cm}
        \item Let $C^{LP}=(C_1^{LP},\dots,C_n^{LP})$ be some optimal solution of (3).
    \end{itemize}
\end{frame}

\section{LP-Based Approximation}
\begin{frame}{LP-Based Approximation}
    \begin{itemize}
        \justifying
        \item If the LP captures sufficient structure of the original problem, the ordering of jobs according to non-decreasing values of $C_j^{LP}$ is a promising candidate for an arbitrary priority policy $\Pi$ and also to a dual guarantee for the quality of the LP lower bound.
        \vspace{0.2cm}
        \item \textbf{Our goal} is to build a proper polyhedron $\pmb{\mathscr{P}}$ and prove the following performance guarantee of $\pmb{\alpha}$ for $\Pi$:
        \vspace{0.2cm}
        \begin{equation}
            \sum\limits_{j\in J}w_jC_j^{LP} \leq Z^{OPT} \leq \sum\limits_{j\in J}w_jE[C_j^{\Pi}]
        \end{equation}
        \begin{equation}
            \sum\limits_{j\in J}w_jE[C_j^{\Pi}] \leq \alpha Z^{OPT} 
        \end{equation}
        %\begin{equation}
        %     \frac{1}{\alpha}Z^{OPT} \leq \sum\limits_{j\in J}w_jC_j^{LP}
        %\end{equation}
    \end{itemize}
\end{frame}

\section{Valid Inequalities for Stochastic Scheduling}
\begin{frame}{Valid Inequalities for Stochastic Scheduling}
    \begin{itemize}
        \justifying    
        \item[(a)] Shultz(1996) 
        \item[] In \textbf{deterministic scheduling}, the following inequality is valid for any feasible schedule on $m$ machines:
        \vspace{0.8cm}
        \begin{equation}
        \sum\limits_{j\in A}p_jC_j \geq \frac{1}{2m} \Bigg(  \sum\limits_{j\in A}p_j\Bigg)^2 + \frac{1}{2} \sum\limits_{j\in A}p_j^2 \qquad \text{for all } A \subseteq J.  
        \end{equation}
        \vspace{0.8cm}
        \item Derived from the \textbf{Q-set constraint(Queyranne, 1993)}.
    \end{itemize}
\end{frame}

\section{Valid Inequalities for Stochastic Scheduling}
\begin{frame}{Valid Inequalities for Stochastic Scheduling}
    \begin{itemize}
        \justifying    
        \item[(b)] Let $\Pi$ be any policy for \textbf{stochastic parallel machine scheduling}. Then the following inequalities are valid for the corresponding vector of expected completion times $E[\pmb{C}^\Pi]$ :
        \vspace{0.4cm}
        \begin{equation}
        \begin{split}
        \sum\limits_{j\in A}E[\pmb{p}_j]E[\pmb{C}_j^\Pi] \geq \frac{1}{2m} \Bigg(  \sum\limits_{j\in A}E[\pmb{p}_j]\Bigg)^2 + \frac{1}{2} \sum\limits_{j\in A}E[\pmb{p}_j]^2 
        \\ - \frac{m-1}{2m}\sum\limits_{j\in A}Var[\pmb{p}_j] 
        \qquad \text{for all } A \subseteq J. \end{split}
        \end{equation}
        \item Due to the \textbf{non-anticipative character} of policies and since processing times are independent, the random variables for the processing time($p_j$) and the start time($S_j^\Pi$) of any job $j$ are stochastically independent.
    \end{itemize}
\end{frame}

\section{Valid Inequalities for Stochastic Scheduling}
\begin{frame}{Valid Inequalities for Stochastic Scheduling}
    \begin{itemize}
        \justifying    
        \item[(c)] With an additional assumption on the 2nd moments of all processing time distributions, we can rewrite (5):
        \vspace{0.4cm}
        \begin{equation}
            \begin{split}
            \sum\limits_{j\in A}E[\pmb{p}_j]E[\pmb{C}_j^\Pi] 
            \geq \frac{1}{2m} \Bigg(\Bigg(  \sum\limits_{j\in A}E[\pmb{p}_j]\Bigg)^2 + \sum\limits_{j\in A}E[\pmb{p}_j]^2 \Bigg)
            \\ - \frac{(m-1)(\Delta-1)}{2m}\sum\limits_{j\in A}E[\pmb{p}_j]^2
            \qquad \text{for all } A \subseteq J. 
            \end{split}
        \end{equation}
        \item[] Assuming $Var[\pmb{p}_j]/E[\pmb{p}_j]^2 \leq \Delta$, \quad for all $A \subseteq J$.
        \vspace{0.8cm}
        \item[(d)] Trivially,
        \begin{equation}
        E[\pmb{C}_j^{\Pi}]\geq r_j+E[\pmb{p}_j] \quad \text{for all } j \in J.
        \end{equation}
    \end{itemize}
\end{frame}

\section{Parallel Machine Scheduling with Release Dates}
\begin{frame}{Parallel Machine Scheduling with Release Dates}
    \begin{itemize}
        \justifying
        \item Lemma 4.1) Let $\Pi$ be a job-based priority policy that schedules the jobs in the order $1<\dots<n$. Then,
        \vspace{0.6cm}
        \begin{equation}
        E[\pmb{C}_j^\Pi] \leq \max\limits_{k=1,\dots,j}r_k+\frac{1}{m}\Bigg( \sum\limits_{k=1}^{j-1}E[\pmb{p}_k] \Bigg)+E[\pmb{p}_j]
        \quad \text{for all }j\in J.
        \end{equation}
        \vspace{0.6cm}
        \item In the absence of release dates, $\max r_k=0$, and the claim also holds for ordinary priority policies.
    \end{itemize}
\end{frame}

\section{Parallel Machine Scheduling with Release Dates}
\begin{frame}{Parallel Machine Scheduling with Release Dates}
    \begin{itemize}
        \justifying
        \item Lemma 4.2) Let $m \geq 1$ and $C\in \mathbb{R}^n$ be any point that satisfies $C_j \geq E[\pmb{p}_j]$ for all $j\in J$ as well as inequality (9) for some $\Delta \geq 0$. Assume $C_1\geq\dots\geq C_n$, then
        \vspace{0.6cm}
        \begin{equation}
        \frac{1}{m}\sum\limits_{k=1}^{j}E[\pmb{p}_k]\leq
        \Bigg( 1+\max \Bigg\{ 1,\frac{m-1}{m} \Delta \Bigg\} \Bigg) C_j \quad \text{for all }j\in J
        \end{equation}
        \vspace{0.6cm}
        \item In the absence of release dates, $\max\limits_k (r_k)=0$, and the claim also holds for ordinary priority policies.
    \end{itemize}
\end{frame}

\section{Parallel Machine Scheduling with Release Dates}
\begin{frame}{Parallel Machine Scheduling with Release Dates}
		\begin{equation*}
		\begin{array}{lll}
			\text{Minimize}& \sum\limits_{j} w_{j}C_j\\[8pt]
            \text{subject to}
            &\sum\limits_{j\in A}E[\pmb{p}_j]E[\pmb{C}_j^\Pi] 
            \geq \dfrac{1}{2m} \Bigg(\Bigg(  \sum\limits_{j\in A}E[\pmb{p}_j]\Bigg)^2 + \sum\limits_{j\in A}E[\pmb{p}_j]^2 \Bigg)\\[20pt]
            & \qquad\qquad\quad - \dfrac{(m-1)(\Delta-1)}{2m}\sum\limits_{j\in A}E[\pmb{p}_j]^2
            \quad \text{for all } A \subseteq J. \\[30pt]
			& E[\pmb{C}_j^{\Pi}]\geq r_j+E[\pmb{p}_j] \quad \text{for all } j \in J.\\[8pt]
		\end{array}
		\end{equation*}
    \begin{itemize}
        \justifying
        \vspace{0.4cm}
        \item Let $C^{LP}$ denote an optimum solution to this LP.
    \end{itemize}        
\end{frame}

\section{Parallel Machine Scheduling with Release Dates}
\begin{frame}{Parallel Machine Scheduling with Release Dates}
    \begin{itemize}
        \justifying
        \item Theorem 4.1)
        \item[] Let $Var[\pmb{p}_j]/E[\pmb{p}_j]^2 \leq \Delta$ for all jobs $j$ and some $\Delta \geq0$, and let $\Pi$ be the job-based priority policy corresponding to an optimal solution to the above LP relaxation. Then, $\Pi$ is a $\pmb{(3-(1/m)+\max\{1,((m-1)/m)\Delta\})}$-approximation.
        \vspace{0.4cm}
        \item Same guarantee can be applied to the without-release-date stochastic scheduling problem. However, we can have a better bound.
        \vspace{3cm}
    \end{itemize}
\end{frame}


\section{Parallel Machine Scheduling without Release Dates}
\begin{frame}{Parallel Machine Scheduling without Release Dates}
    \begin{itemize}
        \justifying
        \item Theorem 4.2)
        \item[] Let $Var[\pmb{p}_j]/E[\pmb{p}_j]^2 \leq \Delta$ for all jobs $j$. Then the \textit{WSEPT} priority policy is a $\pmb{(1+((\Delta-1)(m-1)/2m)}$-approximation.
        \vspace{5.4cm}
    \end{itemize}
\end{frame}

\section{Parallel Machine Scheduling without Release Dates}
\begin{frame}{Parallel Machine Scheduling without Release Dates}
    \begin{itemize}
        \justifying
        \item Theorem 4.2)
        \item[] Let $Var[\pmb{p}_j]/E[\pmb{p}_j]^2 \leq \Delta$ for all jobs $j$. Then the \textit{WSEPT} priority policy is a $\pmb{(1+((\Delta-1)(m-1)/2m)}$-approximation.
        \vspace{0.8cm}
    \end{itemize}
		\begin{equation*}
		\begin{array}{lll}
			\text{Minimize}& \sum\limits_{j} w_{j}C_j\\[8pt]
            \text{subject to}
            &\sum\limits_{j\in A}E[\pmb{p}_j]E[\pmb{C}_j^\Pi] 
            \geq \dfrac{1}{2m} \Bigg(\Bigg(  \sum\limits_{j\in A}E[\pmb{p}_j]\Bigg)^2 + \sum\limits_{j\in A}E[\pmb{p}_j]^2 \Bigg)\\[20pt]
            & \qquad\qquad\quad - \dfrac{(m-1)(\Delta-1)}{2m}\sum\limits_{j\in A}E[\pmb{p}_j]^2
            \quad \text{for all } A \subseteq J. \\[8pt]
		\end{array}
		\end{equation*}    
        \vspace{1.3cm}    
\end{frame}


\end{document}
